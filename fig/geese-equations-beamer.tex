\documentclass[9pt]{beamer}

\input{../notation-thesis.tex}
\begin{document}
	\frame{
		\frametitle{Likelihood}
		\begin{equation}
			\Prcond{\aphyloObs_n}{x_n} = %
			\sum_{\mathbf{x}^n}\Prcond{\mathbf{x}^n}{x_n}%
			\prod_{i \in \offspring{n}}\Prcond{\aphyloObs_i}{x_i};\label{eq:prob-recursive}
		\end{equation}
	
	\begin{equation}
		\label{eq:transition-probability}\Prcond{\mathbf{x}^n=\mathbf{x}}{\ann{n}} = \frac{\exp{\t{\theta} s(\mathbf{x},\ann{n})}}{\sum_{\mathbf{x}^n}\exp{\t{\theta} s(\mathbf{x}^n, \ann{n})}}
	\end{equation}
	where $\mathbf{x}^n \equiv \{x_i^n\}_{i\in\offspring{n}}$ is an array of size $P$ (functions) $\times$ $|\offspring{n}|$ (offspring) representing the state of node $n$'s offspring, $\ann{n}$ is a binary vector representing the state of node $n$, $\theta$ is a column vector of parameters, and $s(\cdot)$ is a column vector of sufficient statistics which may include terms such as: the total number of functional gains, the number of subfunctionalization or neofunctionalization events, etc
	
	}
	\frame{
		\frametitle{Prediction}
		\begin{equation}
			\notag\Prcond{\mathbf{x}^p = \mathbf{x}}{\aphyloObs}  = %
			 \underbrace{\left\{\prod_{m\in\offspring{p}}\Prcond{\aphyloObs_m}{x_m}\right\}}_{\mbox{Everything below }\mathbf{x}^p} %
			\underbrace{\sum_{x_p}\Prcond{x_p}{\aphyloObs}\frac{%
					\Prcond{\mathbf{x}^p = \mathbf{x}}{x_p}%
				}{%
					\Prcond{\aphyloObs_p}{x_p}
			}}_{\mbox{Everything above }\mathbf{x}^p}
		\end{equation}
	}
\frame{
	\begin{equation}
		\Prcond{x_{nk}^p = 1}{x_{pk} = 0, x_{-n}} = \mbox{logistic}\left(\transpose{\Theta}\Delta\chng{nk}\right)
	\end{equation}
}
\end{document}